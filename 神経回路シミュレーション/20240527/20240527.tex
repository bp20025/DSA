%%%%%%%%%%%%%%%%%%%%%%%%%%%%%%
% 週報フォーマット
% 神経情報システム研究室
%%%%%%%%%%%%%%%%%%%%%%%%%%%%%%
\documentclass[dvipdfmx, A4j, twocolumn, 10.5pt]{jsarticle}
% \usepackage[doublespacing]{setspace} % ダブルスペースにしたいときはコメントを外す
\usepackage[margin=20truemm]{geometry} % 上下左右の余白は2cm
\usepackage{graphicx} % 図を挿入
\setlength{\columnsep}{2zw} % 列の間の空白

\usepackage{amsmath} % 数式?
% \usepackage{caption} % キャプションの追加
% \usepackage{biblatex} % 参考文献を扱うパッケージ
\usepackage{comment} % コメントを挿入する

% \addbibresource{references.bib} % リソースを取得


\begin{document}
%\thispagestyle{empty} % ページ番号はいらない

%%% タイトルなど
\twocolumn[%

\centering % 中央寄せ
{\fontsize{18pt}{18pt}\selectfont 週報}\\ % 和文題目
{\fontsize{12pt}{12pt}\selectfont 平岡立成}
\vskip\baselineskip % 一行空け
] 

%%% 本文
\section{今週やったこと}
\begin{itemize}
 \item 『神経回路シミュレーション』山崎匡 読み進め

\end{itemize}
\section{『神経回路シミュレーション』要約}

\section*{第1章 計算神経科学入門}
\subsection*{1.1 計算神経科学とは何か}
神経科学とは,医学,生物学,化学,心理学,情報科学等の集合体で,その中でも理論神経科学は,特に理論的な研究を主とする神経科学であり,数式等への抽象化をすることで脳のモデルを開発する.計算科学のアプローチをとるものをとくに計算神経科学と呼ぶが,理論神経科学と計算神経科学に明確な区別はない.とくに数値シミュレーションに特化した神経科学を「シミュレーション神経科学」と呼び,本書ではこれを扱う.

\subsection*{1.2 神経回路シミュレーション}
神経回路シミュレーションとは,脳の神経回路の挙動を計算機上で数値的にシミュレーションするもののことをいう.

\subsection*{1.3 ニューロンのモデル}
ニューロン(神経細胞)は,樹状突起,細胞体,軸索からなり,パラメータとして膜電位\footnote{細胞外を基準とした細胞内の電位}を持つ.膜電位のダイナミクスは次のように記述できる.

\begin{align*}
C \frac{dV}{dt} = -\bar{g}_{\text{leak}} \left( V(t) - E_{\text{leak}} \right) + I_{\text{ext}}(t)
\end{align*}

これだけでは膜電位が上下するのみでスパイクは発射されない.膜電位は外部からの入力により値が変化し,ある閾値を超えるとスパイク\footnote{短い電気パルス.}を発射する.具体的には,スパイクは,Na\textsuperscript{+}イオンやK\textsuperscript{+}イオンが軸索の表面にあるチャネル\footnote{表面にあるイオンの通り道.刺激に応じて開閉をする.これを電位依存性という.}を通過することで生じる電流により生成する.それぞれの電流は次のように記述できる.

\begin{align*}
I_{\text{Na}}(t) &= -g_{\text{Na}}(V,t)(V(t) - E_{\text{Na}}) \\
I_{\text{K}}(t) &= -g_{\text{K}}(V,t)(V(t) - E_{\text{K}})
\end{align*}

$g_{\text{Na}}(V,t)$,$g_{\text{K}}(V,t)$は,Na\textsuperscript{+}チャネルとK\textsuperscript{+}チャネルのコンダクタンス\footnote{抵抗Rの逆数.電流の流れやすさを示す.}であり,時間と膜電位の関数となる.$E_{\text{Na}}$,$E_{\text{K}}$,は各イオンチャネルの反転電位である.

\begin{figure}[h]
 \centering
 \includegraphics[width=0.45\textwidth]{spike.png}
 \caption{スパイク発生の流れ} 
\end{figure}


\begin{figure}[h]
 \centering
 \includegraphics[width=0.45\textwidth]{actpot.jpg}
 \caption{静止電位と活動電位(スパイク) \cite{lite1}} 
\end{figure}

コンダクタンスの変化により,膜電位の値が急速に上下し,スパイクが生成される.コンダクタンスの式は動物やニューロンの種類により様々であり,世界初のコンダクタンスの式がヤリイカのニューロンを対象としたHodgkin-Huxley方程式である.


コンダクタンスの変化により,膜電位の値が急速に上下し,スパイクが生成される.コンダクタンスの式は動物やニューロンの種類により様々であり,HodgkinとHuxleyはヤリイカのニューロンにおけるスパイク発射のメカニズムを明らかにし,世界で初めてコンダクタンスの式を記述した.

\subsection*{1.4 シナプスのモデル}
スパイクが末端のシナプスへ到達すると,シナプスから神経伝達物質が放出される.放出される神経伝達物質は,ニューロンの種類により異なる.
\begin{itemize}
 \item 興奮性ニューロン(シナプス)... グルタミン酸
 \item 抑制性ニューロン(シナプス)...GABA(γアミノ酪酸)
\end{itemize}

% \printbibliography

これが他のニューロンの樹状突起へ到達すると電流が流れる\footnote{これをシナプス電流という.}.興奮性シナプスからは,脱分極させる電流\footnote{興奮性後シナプス電流,EPSC},抑制性シナプスからは過分極させる電流\footnote{抑制性後シナプス電流,IPSC}がそれぞれ発生する.

\begin{figure}[h]
 \centering
 \includegraphics[width=0.45\textwidth]{isyn.pdf}
 \caption{各シナプス電流による,シナプス電位発生の様子}
\end{figure}

シナプス電流$I_{\text{syn}}$も,シナプスコンダクタンス$g_{\text{syn}}(t)$を用いて次のようにかける.

\begin{align*}
 I_{\text{syn}}(t) = -g_{\text{syn}}(t) \left( V(t) - E_{\text{syn}} \right)
\end{align*}


$E_{\text{syn}}$はシナプスの反転電位\footnote{興奮性シナプスでは0mV,抑制性シナプスでは-65~80mV.}を示す.


\section*{第2章 常微分方程式の数値解法}

\subsection*{2.1 常微分方程式の初期値問題}
$x(t)$を時間tに関する変数とし,その時間変化が$f(x, t)$で与えられるとする.

\begin{equation}
  \frac{dx}{dt} = f(x, t)
\end{equation}
\begin{equation}
  x(0) = x_0
\end{equation}


2つ目の式はxの時刻t=0における値で,初期条件という.2つの式は,ただ1つの自由変数tを持つ.このような微分方程式を常微分方程式と呼ぶ.先程のニューロンの膜電位の式もこの形をとる.これを数値的に解く方法として,オイラー法,ホイン法,ルンゲクッタ法\footnote{こういった方法のことを数値解法という}などがある.

\subsection*{2.2 オイラー法}
常微分方程式を数値的に解く最も簡単な方法がオイラー法(Euler法)である.数学的に微分の定義は

\begin{equation}
  \frac{dx}{dt} = \lim_{\Delta t \to 0} \frac{x(t + \Delta t) - x(t)}{\Delta t}
\end{equation}

であるが,コンピュータ上で無限に小さい値を扱うことはできないので,代わりに十分小さい値$\Delta t$を用いて

\begin{equation}
  \frac{dx}{dt} \approx \frac{x(t + \Delta t) - x(t)}{\Delta t}
\end{equation}


と近似する.この式を変形すると,

\begin{equation}
 x(t+\Delta t)=x(t)+\Delta t f(x, t)
\end{equation}


が得られる.オイラー法は簡単な手法だが精度の悪い方法である.$x(t+\Delta t)$を$t$周りでテイラー展開すると,

$$
\begin{aligned}
x(t+\Delta t) & =x(t)+\frac{1}{1!} f(x, t) \Delta t+\frac{1}{2!} f^{\prime}(x, t) \Delta t^2+\cdots \\
& =x(t)+\frac{1}{1!} f(x, t) \Delta t+O\left(\Delta t^2\right)
\end{aligned}
$$

となる.2式を比較すると,オイラー法は,1次の項までが一致しており,2次以降の後を無視した計算法であることがわかる.オイラー法の誤差はO記法により,O($\Delta t$)と記述することができる.これは$\Delta t$の値を1/100倍すると,誤差も1/100倍となることを意味する.

\subsection*{2.3 ホイン法,ルンゲクッタ法}
その他の方法の詳細な説明は割愛するが,ホイン法の場合,その誤差はO($\Delta t^2$)となる.これは,$\Delta t$が1/100倍されれば,その誤差は1/10000倍にまで減少できることを意味する.
さらに,(4次の)ルンゲクッタ法においてはその誤差はO($\Delta t^3$)まで減少し,さらに精度が高くなることがわかる.

\section*{第3章 神経回路シミュレーション入門}

ニューロンとシナプスの微分方程式による記述と,常微分方程式の数値解法を説明したところで,実際にシミュレーションのプログラムを作成していく.

\subsection*{3.1 ホジキン・ハクスレーモデルのシミュレーション}
単一ニューロンの世界で最初の数理モデルは,Alan L. HodkinとAndrew F. Huxleyにより実現された.
彼らはヤリイカの巨大軸索からの電気生理記録により詳細な解析を行い,Na\textsuperscript{+}イオンとK\textsuperscript{+}イオンによる電流,特に電位依存性コンダクタンスと呼ばれる仕組みが重要な役割を担っていることを明らかにした.この,いわゆるホジキン・ハクスレーモデルが,あらゆるニューロンモデルの基礎となっている.HHモデルは次式で表現される.(次週)


\end{document}
