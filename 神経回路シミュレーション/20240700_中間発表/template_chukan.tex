%%%%%%%%%%%%%%%%%%%%%%%%%%%%%%
% 卒論中間発表予稿フォーマット
% 神経情報システム研究室
%%%%%%%%%%%%%%%%%%%%%%%%%%%%%%
\documentclass[dvipdfmx, A4j, twocolumn, 10.5pt]{jsarticle}
% \usepackage[doublespacing]{setspace} % ダブルスペースにしたいときはコメントを外す
\usepackage[margin=20truemm]{geometry} % 上下左右の余白は2cm
\usepackage{graphicx} % 図を入れるなら
\setlength{\columnsep}{2zw} % 列の間の空白

\begin{document}
\thispagestyle{empty} % ページ番号はいらない
%%% タイトルなど
\twocolumn[%
\centering % 中央寄せ
{\fontsize{18pt}{18pt}\selectfont 大脳新皮質のニューロンにおける奇妙な応答の原因の究明}\\ % 和文題目
{\fontsize{12pt}{12pt}\selectfont The investigation of the causes of peculiar responses in neurons of the neocortex.}\\ % 英文題目
\vskip\baselineskip % 一行空け
{\fontsize{12pt}{12pt}\selectfont 神経情報システム研究室 bp20025 平岡立成 指導教員 保坂亮介准教授} % 学籍・氏名
\vskip\baselineskip % 一行空け
] 
%%% 本文
\section{はじめに}
% 一般的な背景

神経系の基本的な機能単位はニューロンと呼ばれる細胞にあり,単一のニューロンは,隣接するニューロンにより電流が入ると,細胞表面のイオンチャネル等の働きにより膜電位を急激に上昇・下降させることでスパイク発射させる.そのスパイクの発射タイミングやスパイク間隔が,脳が行っている情報処理の解明において重要なファクターとなる.

% 学術的な背景
その中でも,大脳新皮質のニューロンは不規則なスパイク列を生成することが知られている.その不規則なスパイク列の原因として,次の2つが考えられる.

\begin{itemize}
 \item ニューロンの持つスパイク生成メカニズムが不規則なスパイク列を生成している
 \item ニューロンへの入力が不規則なために,出力であるスパイク列も不規則となっている
\end{itemize}

ニューロンが,隣接する十分大きな個数のニューロンから互いに独立なスパイク列を入力として受け取ると,その総和は白色ガウスノイズで近似することができる(Tuckwell, Introduction to Theoretical Neurobiology, 1988).
\cite{lite1}。

% 過去の研究
Sakaiらは,単一ニューロンの原初の数理モデルであるHodgkin=huxleyモデルを用いた研究で,入力の分散が大きくなるほど,出力スパイク列のスパイク間隔の分散が小さくなる奇妙な現象を報告している(Sakai et al, UCNN, 2002).Hogkin-Huxleyモデルは,実際の実験データに忠実に基づいたモデルである分,その複雑さから,Leaky integrate-and-fireモデルやFitzHugh-Nagumoモデルを始め,簡略化されたモデルがいくつか存在する.この奇妙な現象は,Leaky integrate-and-fireモデルにおいては見られず,入力の分散が大きくなるほど,出力スパイク列のスパイク間隔の分散が大きくなった(Sakai et al, IJCNN, 2002).これに対し,HosakaらはHindmarsh-Roseモデルを用いて,この奇妙な現象の原因が,ニューロンの双安定性に起因することを指摘している(Hosaka and Sakai, Physical Review E, 2015).

先にいくつか上げたニューロンの数理モデルは,Hodgkin-Hulxeyモデルをはじめとする.イオンチャネルのダイナミクスを考慮したコンダクタンスベースと,考慮しないカレントベースのモデルに区分できるなど,数理モデルごとに,数式として陽に記述する機能の範囲に差が存在する.Leaky Integrate-and-fireモデルやHindmarsh-Roseモデルはカレントベースのモデルであり,イオンチャネルとの対応が存在しない.したがって,Hodgkin-Huxleyモデルで見られた奇妙な現象の理由が双安定以外にある可能性が否定できません(<- イオンチャネル等がこの奇妙な現象の原因として関わっている可能性がある,って言う趣旨の文章?)


そこで本研究では,以下の段階により,この奇妙な応答が起こる原因を確認することを目的とします.

\cite{lite2}。

\section{奇妙な応答の確認}
先行研究で示されている奇妙な応答を再現します.モデルには一般的なパラメータのHodgkin-Huxleyモデルを用います.入力は,平均と分散という2つのパラメータで表現可能な白色ガウスノイズとします.入力の分散を徐々に変化させていき,その際のニューロンのスパイク列のスパイク間隔の統計量を計算します.この際,スパイク間隔の平均値が一定になるよう入力の平均も分散と同時に変化させます.スパイク間隔の変動係数(Coefficient Variation,CV)を計算し,入力の分散とともに図示することで奇妙な応答の再現を確認します.

\vspace{\baselineskip}

\section{I-F関係と奇妙な応答の確認}
入力電流の変化に応じた発火率の変動(I-F関係)に着目すると,ニューロンは3タイプに分類することができます(Izhikevich, Dynamical systems in neuroscience, 2007).Type2のニューロンは,入力電流の増加に対し段階的に発火頻度を増加させる性質を持ちます.一般的なパラメータのHodgkin-Huxleyモデルは,Type2の応答を持ちますが,$K^{+}$イオンの過渡電流を実装に加えることにより発火頻度を減少させ,Type1の応答を得ることができます.研究2では,このType1のHodgkin-Huxleyモデルが奇妙な応答を示すかを調べます.

\vspace{\baselineskip}
\section{種々のイオンチャネルの影響}
Hodgkin-Huxleyモデルは細胞表面に存在する$Na^{+}$イオンチャネルや$K^{+}イオンチャネルのダイナミクスを記述しますが,実際の細胞には他にも多種のイオンチャネルが存在し,それらを実装に組み込んだ場合のニューロンの応答を調べます.このような,考慮に含まれていなかった機能を実装した際の応答を観察し,この現象の要因の解明を目指します.

\iffalse


\begin{itemize}
 \item ToDoや疑問点
 \item 基本的な内容は研究計画書のままでOKでは.
 \item ->研究計画書に記載されている研究計画は,どれくらいの期間を要して取り組む規模感のものなのか?
 \item 
\end{itemize}

% 問題点
〇〇〇〇〇〇〇〇〇〇〇〇〇〇〇〇〇〇〇〇〇〇〇〇〇〇〇〇〇〇〇〇〇〇〇〇〇〇〇〇〇
〇〇〇〇〇〇〇〇〇〇〇〇〇〇〇〇〇〇〇〇〇〇〇〇〇〇〇〇〇〇〇〇〇〇〇〇〇〇〇〇〇

% 本研究での解決策
〇〇〇〇〇〇〇〇〇〇〇〇〇〇〇〇〇〇〇〇〇〇〇〇〇〇〇〇〇〇〇〇〇〇〇〇〇〇〇〇〇
〇〇〇〇〇〇〇〇〇〇〇〇〇〇〇〇〇〇〇〇〇〇〇〇〇〇〇〇〇〇〇〇〇〇〇〇〇〇〇〇〇

% 結果の概要
〇〇〇〇〇〇〇〇〇〇〇〇〇〇〇〇〇〇〇〇〇〇〇〇〇〇〇〇〇〇〇〇〇〇〇〇〇〇〇〇〇

\section{方法}
〇〇〇〇〇〇〇〇〇〇〇〇〇〇〇〇〇〇〇〇〇〇〇〇〇〇〇〇〇〇〇〇〇〇〇〇〇〇〇〇〇
〇〇〇〇〇〇〇〇〇〇〇〇〇〇〇〇〇〇〇〇〇〇〇〇〇〇〇〇〇〇〇〇〇〇〇〇〇〇〇〇〇
〇〇〇〇〇〇〇〇〇〇〇〇〇〇〇〇〇〇〇〇〇〇〇〇〇〇〇〇〇〇〇〇〇〇〇〇〇〇〇〇〇
〇〇〇〇〇〇〇〇〇〇〇〇〇〇〇〇〇〇〇〇〇〇〇〇〇〇〇〇〇〇〇〇〇〇〇〇〇〇〇〇〇
〇〇〇〇〇〇〇〇〇〇〇〇〇〇〇〇〇〇〇〇〇〇〇〇〇〇〇〇〇〇〇〇〇〇〇〇〇〇〇〇〇
〇〇〇〇〇〇〇〇〇〇〇〇〇〇〇〇〇〇〇〇〇〇〇〇〇〇〇〇〇〇〇〇〇〇〇〇〇〇〇〇〇
〇〇〇〇〇〇〇〇〇〇〇〇〇〇〇〇〇〇〇〇〇〇〇〇〇〇〇〇〇〇〇〇〇〇〇〇〇〇〇〇〇
〇〇〇〇〇〇〇〇〇〇〇〇〇〇〇〇〇〇〇〇〇〇〇〇〇〇〇〇〇〇〇〇〇〇〇〇〇〇〇〇〇

\section{結果}
〇〇〇〇〇〇〇〇〇〇〇〇〇〇〇〇〇〇〇〇〇〇〇〇〇〇〇〇〇〇〇〇〇〇〇〇〇〇〇〇〇
〇〇〇〇〇〇〇〇〇〇〇〇〇〇〇〇〇〇〇〇〇〇(図\ref{fig:matrix})。
〇〇〇〇〇〇〇〇〇〇〇〇〇〇〇〇〇〇〇〇〇〇〇〇〇〇〇〇〇〇〇〇〇〇〇〇〇〇〇〇〇
〇〇〇〇〇〇〇〇〇〇〇〇〇〇〇〇〇〇〇〇〇〇〇〇〇〇〇〇〇〇〇〇〇〇〇〇〇〇〇〇〇
〇〇〇〇〇〇〇〇〇〇〇〇〇〇〇〇〇〇〇〇〇〇〇〇〇〇〇〇〇〇〇〇〇〇〇〇〇〇〇〇〇
〇〇〇〇〇〇〇〇〇〇〇〇〇〇〇〇〇〇〇〇〇〇〇〇〇〇〇〇〇〇〇〇〇〇〇〇〇〇〇〇〇
〇〇〇〇〇〇〇〇〇〇〇〇〇〇〇〇〇〇〇〇〇〇〇〇〇〇〇〇〇〇〇〇〇〇〇〇〇〇〇〇〇

\begin{figure}[htbp] % htbp は置きたい場所によって変える
\includegraphics[width=.48\textwidth]{./example.pdf}
\caption{真の結合行列(左)と推定結果(中央、右)} \label{fig:matrix}
\end{figure}


\section{考察}
% 目的の要約
〇〇〇〇〇〇〇〇〇〇〇〇〇〇〇〇〇〇〇〇〇〇〇〇〇〇〇〇〇〇〇〇〇〇〇〇〇〇〇〇〇

% 結果の要約
〇〇〇〇〇〇〇〇〇〇〇〇〇〇〇〇〇〇〇〇〇〇〇〇〇〇〇〇〇〇〇〇〇〇〇〇〇〇〇〇〇

% この研究の結果で問題は解決したのか?
〇〇〇〇〇〇〇〇〇〇〇〇〇〇〇〇〇〇〇〇〇〇〇〇〇〇〇〇〇〇〇〇〇〇〇〇〇〇〇〇〇

% 結果から新たに分かったこと、新たな問題
〇〇〇〇〇〇〇〇〇〇〇〇〇〇〇〇〇〇〇〇〇〇〇〇〇〇〇〇〇〇〇〇〇〇〇〇〇〇〇〇〇

% 今後の課題
〇〇〇〇〇〇〇〇〇〇〇〇〇〇〇〇〇〇〇〇〇〇〇〇〇〇〇〇〇〇〇〇〇〇〇〇〇〇〇〇〇

\fi

\begin{thebibliography}{99}
\bibitem{lite1} 保坂亮介, 木村貴幸, 松浦隆文 ``タイトル,'' 雑誌名, pp.10-13, 2022.
\bibitem{lite2} R. Hosaka, T. Kimura, and T. Matsuura, ``title,'' journal name, pp.10-20, 2021.
\end{thebibliography}
\end{document}
