%%%%%%%%%%%%%%%%%%%%%%%%%%%%%%
% 週報フォーマット
% 神経情報システム研究室
%%%%%%%%%%%%%%%%%%%%%%%%%%%%%%
\documentclass[dvipdfmx, A4j, twocolumn, 10.5pt]{jsarticle}
% \usepackage[doublespacing]{setspace} % ダブルスペースにしたいときはコメントを外す
\usepackage[margin=20truemm]{geometry} % 上下左右の余白は2cm
\usepackage{graphicx} % 図を挿入
\setlength{\columnsep}{2zw} % 列の間の空白

\usepackage{amsmath} % 数式?
% \usepackage{caption} % キャプションの追加
% \usepackage{biblatex} % 参考文献を扱うパッケージ
\usepackage{comment} % コメントを挿入する

% \addbibresource{references.bib} % リソースを取得


\begin{document}
%\thispagestyle{empty} % ページ番号はいらない

%%% タイトルなど
\twocolumn[%

\centering % 中央寄せ
{\fontsize{18pt}{18pt}\selectfont 週報}\\ % 和文題目
{\fontsize{12pt}{12pt}\selectfont 2024/07/01} \\
{\fontsize{12pt}{12pt}\selectfont 平岡立成}
\vskip\baselineskip % 一行空け
] 

%%% 本文
\section{今週やったこと}
\begin{itemize}
 \item 志望理由書の作成
 \item 研究計画書の作成
 \item 出願の完了
 \item 『神経回路シミュレーション』山崎匡 読み進め

\end{itemize}
\section{志望理由書・研究計画書の概要}
(PDF参照)



\section{今週やること}
\begin{itemize}
 \item 『神経回路シミュレーション』山崎匡 読み進め(Izhikevichモデル等)
 \item ??
\end{itemize}

\end{document}
